\documentclass[12pt,spanich]{article}
\usepackage[spanish,activeacute]{babel}
\usepackage[latin1]{inputenc}
\oddsidemargin 6in
\textwidth 6.75in
\topmargin 0in
\textheight 8.5in
\parindent 0em
\parskip 1ex
\begin{document} 


\begin{center}
{\bf SISTEMA OPERATIVO ULTRIX\\ }                                  
\end{center}

\section{Generalidades}

\begin{itemize} 
                       \item multiusuario
                       \item multiproceso
\end{itemize}

\subsection{Comandos para manejo de Archivos}

  {\bf Generalidades :}\\  \\ 
Los  siguientes  comandos pueden ser utilizados  tanto
con archivos en el directorio actual, como con
archivos en otros directorios. Los archivos a distancia pueden ser `alcanzados` dando su
`path` (direccion) en el sistema; las direcciones se especifican
de la forma :\\                                                        
\begin{center}
 {\bf /usr/user/usuarios/juperez}\\ 
\end{center}
Es la direccion  de  Juanito  Perez,
este usuario tiene definido su path en el caracter %\verb'~', luego  si  crea
un directorio (ver mas adelante) , y este se  llama XX ,  entonces  la
direccion de los archivos que quiera accesar dentro de este sera:\\ 

\includegraphics{./logo_pinguino.jpg}

\begin{tabular}
    Adams & 0.3 & 1,3 \\
    Merry & T & F  \\
\end{tabular}

\begin{enumerate}
  \item {\bf prt} aa -f xxxx : comando que permite imprimir el archivo a en el 
formato xxxx (doc6, doc8, ofi8, etc).                           

  \item {\bf man}  xxxx      : help del sistema que permite obtener informacion del
                   topico xxxx.                                                 

  \item {\bf apropos} xxxx   : permite  obtener un  listado de topicos relacionados
                   con el string dado xxxx.

  \item {\bf vi} aa          : permite  editar el archivo aa si este existe, o  bien,
                   crearlo si no es asi.

  \item {\bf mail} user verb'<' aa : permite enviar un mail, almacenado previamente en el
                   archivo aa, a el usuario indicado por user que  esta
                   en el mismo computador (nodo) nuestro.
\end{enumerate}
\end{document}